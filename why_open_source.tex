The concept of trust is indelibly woven into the very essence of blockchain technologies. It must be for the reason discussed above - blockchain technologies are used to enable direct, peer-to-peer transactions between parties that don’t fully trust one another or have a trusted central authority. Consequently, trust in the blockchain technologies themselves is an essential prerequisite to their adoption. 

We believe that an open source, collaborative approach that invites participation from all stakeholders is the most effective way to produce the necessary degree of trust in blockchain technologies for businesses to broadly and rapidly adopt them. The transparency of open source development facilitates audit and review from a diverse community of experts. This practice of open development and review is standard within the security and cryptography communities to ensure correctness of concept and implementation.

An effective open source project must have an open governance model, an open development model and an open review model.  Hyperledger fully embraces the transparency and openness common to Linux Foundation Projects, which provides the legal, governance, technical, logistical and promotional structure that all software initiatives need. Anyone can download the codebase and start contributing, and the positions of authority are determined in an open and democratic manner. 

The open source structure has been the driving force of the growing number of distinct software projects, contributors, meet-up groups, hackfests, and corporate members, all under the Hyperledger umbrella.  Companies deploying blockchain internally, and those building products and services based on Hyperledger projects tell us that their trust and confidence comes from knowing that Hyperledger technologies are built in the open, with and by an extremely broad consortium of users and vendors regularly reviewing and checking the code to ensure that it is of the highest standard. 

The open source nature of Hyperledger technologies also ensures no surprises when it comes to integration and interoperability between various blockchains - something we believe will be very common in the poly-chain future we expect. We expect that achieving needed interaction across proprietary chains would otherwise be much more difficult. In addition to interoperability between different chains, the open source nature of Hyperledger will enable more application portability between various blockchain implementations, hopefully leading to easier application creation for developers.

Economics is also powerfully on the side of a collaborative development effort like Hyperledger. Businesses as diverse as banks, car and plane makers, healthcare companies and a broad ecosystem of vendors, all need robust, feature-rich, modular blockchain platforms that they can tailor to meet their exact needs. When all these different users and vendors collaborate to co-create common platform technologies, the investment required from each is a fraction of what it would be if each created their own.
