\subsection{Health Records: Credentialing}

Blockchain technologies promise to reduce one of the great annoyances of
modern medical practice: ``credentialing''. Hospitals use the
credentialing process to make sure that its physicians are competent and
trustworthy. In a sense, credentialing is the hospital's way of
performing ``due diligence'' on a physician.

But today this process imposes a huge burden, both on the physician
applying for affiliation and the hospital that must vet the
applications.

\subsubsection{The physician gathers credentials, many on paper}

Any physician who wishes to become affiliated with a hospital begins the
process by gathering copies of all their professional credentials, such
as:

\begin{itemize}
\item Medical school diploma
\item Certificates of any residencies and fellowships completed
\item Certificates from any specialty medical boards
\item All state medical licenses
\item Evaluations from peers
\item Proof of meeting requirements for continuing medical education
\item Letters from hospitals where the physician was previously
      affiliated, explaining how and why  the affiliation ended
\item Details of any malpractice suits
\end{itemize}

\Mic{This paragraph is rather simplistic. Many of the required documents
are electronic, but simply not available. Some are paper hand must be
converted though often physical artifacts are used. My recommendation is
that we just get rid of this.}
Many of these documents were originally provided on paper, so the
physician may need to scan in paper copies, and manage a set of scanned
images or PDFs.

\subsubsection{The hospital checks credentials and calls a meeting}

On the hospital's side, the credentialing office checks the physician's
documentation for completeness, accuracy, and authenticity.  This is an
exacting task.  Almost inevitably, they find shortfalls and go back to
the doctor for missing documents.

Then the credentialing office verifies some or all of the physician's
documentation.  For example, they may phone a medical school to confirm
that the physician did indeed graduate from there.  This is clearly a
time-intensive process that is prone to errors.

Once the documentation is determined to be complete, accurate, and
authentic, the hospital's credentialing committee---which typically
includes physicians and administrators---meets to decide whether the
physician can begin practicing in affiliation with the hospital.

\Mic{Again, I'm not sure if calling out paper bound is either helpful or
accurate. I would prefer to simply leave it as ``complex''.}
The entire credentialing process is paper-bound, time-consuming, and
low-trust. And it can take weeks or even months until any physician is
cleared to affiliate with a hospital.

\subsubsection{Three key questions for any blockchain solution}

An effective blockchain solution for medical credentialing must answer
three key questions about content, identities, and resources.

\begin{enumerate}
\item Will actual content or only pointers to content be placed on the
blockchain?  Credentialing solutions might place public information
(such as state medical licensing) on the blockchain itself.  However,
private information (such as peer reviews) might be better stored off
the chain; this would guard against any loss of keys, and enable users
to remove---but not change---private information.

\item What's the best way to manage the identities of many participants?
An ambitious credentialing solution might include every hospital, every
physician, every source of continuing medical education, and so on.
This could eventually number thousands of participants.  How will so
many identities be efficiently and securely managed?

\item What resources are required, especially for storage?
Credentialing solutions may be in service for decades, requiring
significant resources for processing, communication, and storage.  For
example, what if at some point credentialing organizations want video
testimony from peers? Storage requirements could skyrocket---and who
would cover that added cost?
\end{enumerate}

\subsubsection{Hyperledger can help streamline credentialing}

Credentialing provides a good use case for blockchain technologies,
which can simplify and streamline every step of the process.

\textbf{Hyperledger Indy} provides off-the-shelf solutions that would
otherwise require architecting and developing new software. One
significant feature: Indy implements the proposed W3C standard for
verifiable claims, supporting the pairwise exchange of selected
credentials. In practice, this can work as follows:

\begin{enumerate}
\item A physician requests proof of graduation from their medical school.

\item The medical school places a digital credential on the blockchain
on behalf of the physician, where it's considered irrefutable and
tamper-proof.

\item A hospital can access the blockchain to verify the physician's
credential, with no need to contact the medical school directly.

\item The physician can expose only the specific credentials the
hospital requires, and nothing more.
\end{enumerate}

This implementation of verifiable claims safeguards the physician's
privacy, saves time and effort for everyone involved, and streamlines
the entire process. At last: a better way to handle medical
credentialing.
