\section{Medical Credentials}

Blockchain technologies have the potential to ameliorate one of the great annoyances of modern
medical practice: ``credentialing''. Credentialing is the process a hospital uses to ensure that
its physicians are competent and worthy of the trust that patients put in them. In a sense,
credentialing is the hospital's way of performing ``due diligence'' on a physician.

A physician who wishes to become affiliated with a hospital begins the process by first gathering
copies of all of his or her professional credentials including, for example:

\begin{itemize}
\item Medical school diploma
\item Certificates of any residencies and fellowships the physician completed
\item Copies of any specialty medical boards that have certified the physician
\item All state medical licenses held by the physician
\item Evaluations from peers
\item Proof that the physician is current on continuing medical education requirements
\item Letters from hospitals with which the physician was previously affiliated, explaining the
  circumstances under which the affiliation ended 
\item Details of any malpractice actions against the physician
\end{itemize}

The hospital's credentialing office checks the physician's documentation for completeness, accuracy,
and authenticity. This is an exacting task.  Almost inevitably, they will find shortfalls, and will
ask the physician to supply missing documents. In many cases, the hospital's credentialing office
will verify some or all of the physician's submitted documentation, e.g., telephoning the
physician's medical school to confirm that the physician did indeed graduate from there.  It is not
uncommon for weeks or months to elapse as the physician and credentialing office work to satisfy the
hospital's requirements.

Once the documentation is determined to be complete, accurate and authentic, the hospital's
credentialing committee, typically composed of both physicians and administrators, sits in judgment
of the physician and decides whether or not to allow the physician to begin practicing in
affiliation with the hospital.

\subsection{Credentialing with a Blockchain}

When using blockchain technology in any solution, several key decisions must be made.

First, will content or pointers-to-content be placed on the blockchain?  For credentialing
solutions, it might be reasonable to place publically available information (such as state medical
licensing) on the blockchain itself.  However, private information (such as peer reviews) are better
stored off the chain to guard against compromise of encryption keys and give users the ability to
remove (but not edit) information, thereby increasing trust.

Second, what is the best way to manage the identities potentially of thousands of participants?  For
ambitious credentialing solutions, this might include every hospital, every physician, every
provider of continuing medical education, and so on.

Third, what are the resource requirements, specifically storage? Credentialling solutions may
provide service for decades. Persistent commitment of participation comes with a potentially
significant contribution of resources for compute, communication, and storage. For example, what if,
a few years from now, credentialing organizations want video testimony from peers?

\subsection{Credentialling with Hyperledger Indy}

Hyperledger Indy provides off-the-shelf solutions for what would otherwise require careful
engineering of new software modules, with all the consequent risks in critical areas such as
lsecurity and scalability.

% -----------------------------------------------------------------
% -----------------------------------------------------------------
% -----------------------------------------------------------------
% -----------------------------------------------------------------
% -----------------------------------------------------------------


%% \subsection{A Simple Solution}

%% The immutability guarantees provided by blockchain technologies appear to be well-suited to improve
%% the efficiency of credentialing. For example, we could construct a blockchain where every piece of
%% documentation and every verification action is committed to the permanent ledger through a
%% transaction.

%% Immutability of the blockchain means that once a document is committed to the ledger, it cannot be
%% changed. It further means that any verification of one of these immutable documents can be
%% considered valid indefinitely. Therefore, once a certain number of verifications have been stored
%% for a document, further verifications would be redundant and perhaps pointless.

%% Skipping redundant verifications could significantly reduce the workload of hospital credentialing
%% offices. However, an even greater benefit would accrue if medical schools, hospital residencies,
%% training fellowships, specialty board examinations, state licensing departments, continuing
%% education providers, peer reviewers, and others put their issued certifications into a blockchain
%% themselves.  This would provide substantial and immediate assurance to hospitals that these
%% components of a physician's professional credentials were inarguably genuine, perhaps eliminating
%% the need to verify these documents at all.

%% \subsection{A Better Solution}

%% The simple solution demonstrates the benefits of using a blockchain for credentialing. However,
%% there are several problems: a blockchain is not the best technology for long term storage of large
%% data files such as the documents necessary for credentialing. Further, transparency of storage opens
%% the door for abuse of privacy and confidentiality.

%% Hypledger Indy offers an alternative approach...

