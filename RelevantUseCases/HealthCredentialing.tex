\subsection{Health Records: Credentialing}

Blockchain technologies have the potential to ameliorate one of the great annoyances of modern
medical practice: ``credentialing''. Credentialing is the process a hospital uses to ensure that
its physicians are competent and worthy of the trust that patients put in them. In a sense,
credentialing is the hospital's way of performing ``due diligence'' on a physician.

A physician who wishes to become affiliated with a hospital begins the process by first gathering
copies of all of his or her professional credentials including, for example:

\begin{itemize}
\item Medical school diploma
\item Certificates of any residencies and fellowships the physician completed
\item Copies of any specialty medical boards that have certified the physician
\item All state medical licenses held by the physician
\item Evaluations from peers
\item Proof that the physician is current on continuing medical education requirements
\item Letters from hospitals with which the physician was previously affiliated, explaining the
  circumstances under which the affiliation ended 
\item Details of any malpractice actions against the physician
\end{itemize}

The hospital's credentialing office checks the physician's documentation for completeness, accuracy,
and authenticity. This is an exacting task.  Almost inevitably, they will find shortfalls, and will
ask the physician to supply missing documents. In many cases, the hospital's credentialing office
will verify some or all of the physician's submitted documentation, e.g., telephoning the
physician's medical school to confirm that the physician did indeed graduate from there.  It is not
uncommon for weeks or months to elapse as the physician and credentialing office work to satisfy the
hospital's requirements.

Once the documentation is determined to be complete, accurate and authentic, the hospital's
credentialing committee, typically composed of both physicians and administrators, sits in judgment
of the physician and decides whether or not to allow the physician to begin practicing in
affiliation with the hospital.

When using blockchain technology in any solution, several key decisions must be made.

First, will content or pointers-to-content be placed on the blockchain?  For credentialing
solutions, it might be reasonable to place publically available information (such as state medical
licensing) on the blockchain itself.  However, private information (such as peer reviews) are better
stored off the chain to guard against compromise of encryption keys and give users the ability to
remove (but not edit) information, thereby increasing trust.

Second, what is the best way to manage the identities potentially of thousands of participants?  For
ambitious credentialing solutions, this might include every hospital, every physician, every
provider of continuing medical education, and so on.

Third, what are the resource requirements, specifically storage? Credentialling solutions may
provide service for decades. Persistent commitment of participation comes with a potentially
significant contribution of resources for compute, communication, and storage. For example, what if,
a few years from now, credentialing organizations want video testimony from peers?

Hyperledger Indy provides off-the-shelf solutions for what would otherwise require careful
engineering of new software modules. Indy implements the proposed W3C standard for verifiable
claims. This capability provides a method for pairwise exchange of selected credential
attributes. For example, a physician could request a credential from their medical school that
attests to their successful graduation. That credential could be provided by the physician to a
hospital as verification of education. An important property of Indy and their implementation of
verifiable claims is that the the credential for education can be verified by the hospital as
graduation from an accredited university without the need to contact the medical school
directly. The applying physician need only expose precisely what is required for credentialing at
the hospital; no additional exposure is necessary.

