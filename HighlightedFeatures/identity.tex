Indy shares some features with traditional enterprise identity solutions--the world of LDAP, OAuth, 2FA, IDPs, and similar tech. Both approaches use industrial-strength crypto. Both enable capturing and sharing rich metadata about an identity. Both facilitate sophisticated access control and policy. But there is a profound difference: Indy identities are shared instead of siloed and federated. An Indy identity is portable--you can bring it with you wherever the distributed ledger is accepted. Ten orgs or systems that each support Indy identities don’t create ten separate identities for John Q. Public; John simply shows up with his pre-existing identity and uses it. Organizations can cancel John’s access, but never his identity, because John owns it. John, not the places that accept John’s identity, controls access to his data.

Indy also shares some features with blockchain-based identity solutions such as Blockstack and Uport. All of these technologies store identity on a distributed ledger and thus promote security and personal freedom. However, Blockstack and Uport depend on Bitcoin and Ethereum, respectively. These are proof-of-work ecosystems that impose a non-trivial cost on transactions; every new persona, public key rotation, published attribute, or pairwise relationship is a tangible expense. This creates a disincentive to pairwise relationships, which undermines privacy. Also, these ecosystems are global and public; they cannot be special-purposed for a less-than-fully-global context. Indy, on the other hand, does not use proof-of-work--transactions are free. And instances of it can be used in whatever context is convenient.
