In this paper, we have explained the rationale behind the creation of Hyperledger and what our goals were, and still are, for the project. We outlined why we think an open source umbrella organization seems to be the optimal governing arrangement for a general blockchain consortium and showcased some of the use cases that inspired our members to join and work on Hyperledger as well as delineated some of the features that result from building blockchains for some of these interesting use cases. In addition, we briefly summarized all of the main Hyperledger projects and their statuses.

We hope that this is just the beginning for many readers. While this paper is not intended to be technical, we note that there is a wealth of technical information on the Hyperledger wiki. Each of the main projects has quite a bit of documentation, getting-started guides, and help available there\footnote{If this isn't true, it should be!}. In addition, some of the working groups have great technical resources. The architecture working group has a substantial amount of documentation on permissioned blockchain fundamentals and is a great resource for those looking to explore technical details. There are also application-specific working groups that are great places to learn. For instance, he identity working group has spent a lot of time discussing and documenting how blockchain can enable identity solutions. We encourage readers to look to these places for more information on topics that they find interesting.

We hope that this paper is just the beginning of the Hyperledger experience for many. We acknowledge that there is a lot of work left to be done, and that Hyperledger will almost always be a work-in-progress. But, we think our organization is solid and believe that, maybe with your help, we can build secure, efficient, and reliable blockchain solutions.
