While blockchain is a powerful piece of technology, it is not a one-size-fits-all solution. Modifications and special features are needed in order for it to operate at an optimal level, or even operate at all for its intended purpose. Customized functionality and features are essential to making blockchain technology the appropriate solution for the organization that uses it. Since organizations have varying needs, it can be assumed that there will be multiple blockchains customized with different features for a wide-range of solutions instead of a single standard blockchain.

Because of this development of multiple systems, collaboration plays a larger role in order to help reduce the overall resources consumed with these efforts. For example, navigating through various developments in an open source environment can be daunting and subsequently cause organizations to forego keeping up with the changes or starting at all due to significant costs. Hyperledger fills this organizational need and makes the coordination process more effective by creating a collaborative environment that streamlines project development and communication. With this environment, keeping up with the various developments in the blockchain industry is simplified. This also makes it easier for newer participants that join the umbrella organization to catch up with the latest developments as information is properly structured to ensure that new participants have an easy way of joining the collaborative effort.

The umbrella organization structure also allows for specialization among the participants. Specialization has historically proven to be a driving factor in global economic development and the same benefits can be realized with participants specializing in various  areas of the technology. Outside an umbrella organization, such would still be possible in an open source environment, but would be much more difficult. As communication and collaboration is streamlined, implementation of these developments and access to necessary information can be done with ease across various projects for the benefit of the entire ecosystem. However, unlike most cases, participants that specialize in similar areas won’t be competing against each other. With an umbrella organization, joint research efforts are not only possible, but also encouraged in order to prevent duplicate efforts, which have a stronger negative effect in a relatively new industry where the development pool is not yet deep.. 

The umbrella organization facilitates more collaboration between industry participants than would otherwise occur.  This can streamline the development of newer projects and avoid duplicative efforts by allowing for the creation of common components that benefit the entire community. Interoperability between ledgers, similar or not, also becomes much more of a possibility, not just because of a better understanding of the other ledger projects, but also because of the  collaborative environment. The governing structure provided by Hyperledger also helps solve potential interoperability disputes that may arise. 

The structure enforces quality control by having a technical governing committee review all projects throughout their life cycles.  For new projects, this provides a chance to be critiqued and gain knowledge from members of all existing projects. Cross-project exposure increases the likelihood of collaboration, which can potentially mitigate duplicated efforts. Additionally, existing project members may discover innovations and developments introduced by the new projects, and implement them in their own projects. This structure also fosters potential interoperability between new and old projects.

Consistency of licenses and handling of intellectual property is another value provided by the umbrella organization. In particular, Hyperledger operates under an Apache 2.0 license for code and Creative Commons Attribution 4.0 International license for content. These licenses are known to be particularly enterprise friendly. A single, consistent approach to intellectual property removes the need for expensive and complex development relationships amongst the members. Expectations for participants are clearly communicated and those building and using Hyperledger technologies can participate without fear of hidden legal encumbrances.
