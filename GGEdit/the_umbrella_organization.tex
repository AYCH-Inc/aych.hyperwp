Hyperledger serves as an umbrella organization that brings together users, developers, and vendors from many different sectors and market spaces. 
All these participants have one thing in common: Everyone is interested in learning about, developing, and using enterprise blockchains. 

~\newline
\emph{Here's a great place for an org chart drawn somewhat like an umbrella.---GG}
~\newline

While blockchain is a powerful technology, it is not one-size-fits-all. 
Every enterprise needs special features and modifications to help a blockchain achieve its intended purpose.
Since different organizations have different needs, there will never be one single, standard blockchain. 
Instead, we expect to see many blockchains customized with different features that provide a wide range of solutions across many industries.

Because enterprises need to understand, develop, and use many different blockchains, an umbrella organization like Hyperledger can help reduce the resources consumed by these efforts.

As the umbrella organization for open source blockchain development, Hyperledger provides these benefits: 
\begin{itemize}
\item Help keeping up with developments
\item Better productivity through specialization
\item Collaboration to avoid duplicate efforts
\item Better quality control of code
\item Easier handling of intellectual property
\end{itemize}

\subsection{Help keeping up with developments}
Navigating through all the developments in an open source environment can be daunting. 
Due to the cost and complexity involved, some organizations may give up, or never get started at all. 

Hyperledger reduces this research burden by creating a collaborative environment that streamlines communication. 
Better communication helps new participants to catch up, by gaining faster access to necessary information.
As newer participants quickly join the collaborative effort, this speeds up development, for the benefit of the entire community. 

\emph{Question: Sounds good, but how? Any concrete details we can offer?---GG}

\subsection{Better productivity through specialization}
A basic premise of economics ever since Adam Smith is that \emph{specialization}---also known as division of labor---leads to higher productivity. 
Instead of everyone doing a little bit of everything, specialization enables people to focus their energies on fewer tasks and become more expert at them. 
The benefits of specialization include more expertise, more value added, and ultimately more wealth created. 
This is why specialization has proven to be a driving factor in global economic development. 

Participants can gain the same benefits---more expertise, more value added, and better all-around productivity---by specializing in certain areas of a new technology like blockchain. 
In an open source environment that lacks any umbrella organization, this would be far more difficult. 

Hyperledger's umbrella structure encourages specialization, which yields better productivity. 
And participants who happen to specialize in similar areas aren't competing against each other. 
In an umbrella organization, specialists are encouraged to join forces to accelerate their research and development. 

\subsection{Collaboration to avoid duplicate efforts}
In a siloed environment, many people can unwittingly duplicate each another's efforts. Duplication of effort is especially negative in a new industry like blockchain, where the talent pool of seasoned developers is not yet deep. 

In an umbrella organization, collaboration between participants is highly encouraged. 
This can avoid duplication, streamline the development of new projects, and encourage the creation of common components that benefit the entire community. 

Interoperability between various distributed ledgers is also enhanced by a better understanding of other projects. 
And the governing structure provided by Hyperledger can help solve any interoperability disputes that could potentially arise. 

\subsection{Better quality control of code}
Open source software is recognized for its high quality, achieved through careful code reviews and significant debugging. 
Hyperledger promotes quality control by having a technical governing committee review all projects throughout their life cycles.  
This gives new projects a chance to be critiqued, so their developers can gain knowledge from all the existing projects. 
For their part, long-time project members may discover innovations in new projects which can enhance their own projects. 
This umbrella structure also fosters interoperability between new and existing projects.

\subsection{Easier handling of intellectual property}
Another benefit provided by the umbrella organization is easier, more consistent handling of intellectual property. 
Hyperledger operates under an Apache 2.0 license for code (see \url{https://www.apache.org/licenses/LICENSE-2.0}) and Creative Commons Attribution 4.0 International license for content (see \url{https://creativecommons.org/licenses/by/4.0/}). 
Both these licenses are known to be particularly enterprise-friendly. 

A single, consistent approach to intellectual property removes the need for complex and expensive contractual   relationships among members. 
Since all participants have clearly communicated their expectations, anyone building and using Hyperledger technologies can participate without fear of running into hidden legal encumbrances.

