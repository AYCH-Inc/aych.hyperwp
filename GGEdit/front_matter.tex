\subsection{About Hyperledger}
Hyperledger is an open source collaborative effort, created to advance blockchain technology by addressing important features for a cross-industry open standard for distributed ledgers. 
It is a global collaboration that includes leaders in banking, finance, Internet of Things, manufacturing, supply chains, and technology. 

The Linux Foundation hosts Hyperledger as a Collaborative Project under the foundation. 
Hyperledger does not promote a single blockchain codebase or a single blockchain project. 
Rather, it enables a worldwide developer community to work together and share ideas, infrastructure, and code. 

\subsection{Purpose of this Paper}
 This paper provides a high-level overview of the Hyperledger project: Why it was created, how it is governed, and what it hopes to achieve. 
 The core of this paper presents five compelling uses for enterprise blockchain in different industries. 
 It also describes the open source frameworks that Hyperledger is developing to help enterprises around the world deliver on the promise of blockchain for more secure, more reliable, and more streamlined interactions. 
 
 This is not intended as a deep technical white paper, but an introduction to Hyperledger for a general business reader. 
 
\subsection{Intended Audience}
We expect this paper will be read by business people from different backgrounds, including entrepreneurs, executives, IT managers, and software developers. 
Since the blockchain is so new, we expect different readers will be more or less familiar with certain blockchain terms and concepts. 
And since Hyperledger is a worldwide project, we expect this paper will be read by people around the world, many of whom do not have English as their first language. 

Therefore, we  tried to make this paper as clear and readable as possible. 
The Further Resources section at the end points to more introductory and more advanced materials you may want to explore. 
  
\subsection{Outline}
This paper covers the following material: 
\begin{itemize}
\item Section 1 introduces the growing need for distributed ledgers in enterprises of today and tomorrow.
\item Section 2 shows why open source is a good fit for enterprise blockchain development.
\item Section 3 describes the umbrella structure of the Hyperledger project and how that leads to good governance.
\item Section 4 defines the key elements of the design philosophy that all Hyperledger projects must follow.
\item Section 5 presents five compelling uses cases for blockchain from different industries: banking, financial services, healthcare, IT, and supply chains. For many readers, this will be the most intriguing part of this document. 
\item Section 6 outlines each of the current top-level projects in Hyperledger.
\item Section 7 explains the long-term vision for the Hyperledger project.
\item Section 8 offers some final thoughts.
\end{itemize}

\subsection{Acknowledgements}
The Hyperledger White Paper Working Group would like to thank all the following people for contributing to this paper:

Tamas Blummer, Sean Bohan, Mic Bowman, Christian Cachin, Nick Gaski, Nathan George, Gordon Graham, Daniel Hardman, Ram Jagadeesan, Travin Keith, Renat Khasanshyn, Murali Krishna, Tracy Kuhrt, Arnaud Le Hors, Stanislav Liberman, Esther Mendez, Dan Middleton, Hart Montgomery, Dan O'Prey, Drummond Reed, Stefan Teis, Dave Voell, Greg Wallace, Baohua Yang.

We would also like to thank the Hyperledger Technical Steering and Marketing Committees for their valuable feedback throughout the writing of this paper.

