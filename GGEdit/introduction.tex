\textbf{Please note}: I'd like this document to start off with a clear introduction. 
But I'm very concerned that the current Introduction will turn off many business readers.
If we lose them on the first page, we've lost them for good. 
I found this section hard to read, for these reasons:
\begin{itemize}
\item This is supposed to be ``the introduction to the introduction" to Hyperledger---but it's not clear why we start in talking about databases. 
\item It doesn't move from the familiar to the unfamiliar. 
\item It jumps right into a technical discussion, without defining any terms. 
\item This seems too complex for a business reader, but perhaps not deep enough for a technical reader.
\end{itemize} 

I propose revising this section to spell out the following step-by-step logic. 
Please let me know what you think.
Once we agree on a direction, I can write up a draft in a couple of days.
And ideally, we can come up with some diagrams to make this more visual and inviting.
Thank you.---GG\\

\textbf{Databases are everywhere}
\begin{itemize}
\item Familiar example: contacts on a cell phone, electronic version of paper address book
\item Define database
\item Very brief sketch of early history of databases (2 or 3 sentences)
\item Q: Do we need to define hierarchical and relational databases and SQL? (I think not)
\end{itemize}

\textbf{Many databases today are shared}
\begin{itemize}
\item Familiar example: shared calendars, electronic version of paper calendar
\item More elaborate shared databases are used in business
\item Business example: list of inventory by SKU, shared by order desk at home office and by account rep with laptop in the field, both can take orders and allocate stock-on-hand to customers
\end{itemize}

\textbf{But shared databases raise questions}
\begin{itemize}
\item Who can do what, when and how?
\item Who do you trust? 
\item Who settles any conflicts or disputes? 
\item What happens if both the home office and a field rep want to sell the same item? 
\item What happens if both an account rep and an e-commerce shopper want the same item?
\item Familiar example for the double-spend problem 
\item Conclusion: we need clear rules to share a database effectively
\item Touch on replicated and federated databases very briefly 
\item Any other definitions here?
\end{itemize}

\textbf{Blockchain is a new form of shared database}
\begin{itemize}
\item Define blockchain
\item Very brief sketch of history of blockchain (2 or 3 sentences)
\item Define consensus
\item Define trust and trustless
\item Define permissionless and permissioned
\end{itemize}

\textbf{There's more to blockchain than Bitcoin}
\begin{itemize}
\item The media is full of stories on Bitcoin and other crypto-currencies
\item Most businesses don't care about crypto-currencies, and they never will
\item The underlying technology---blockchain---is far more significant for enterprises 
\item Quote Don Tapscott: Blockchain = internet of value, not just information
\item Of course, businesses are interested in value
\end{itemize}

\textbf{Blockchain solves the problems of shared databases}
\begin{itemize}
\item How blockchain solves trust
\item How blockchain solves the double-spend problem
\item With blockchain, many existing business processes can be streamlined
\item And many new processed can be invented
\end{itemize}

\textbf{Hyperledger was created to further blockchain for enterprises}
\begin{itemize}
\item Very brief sketch of history of Hyperledger (2 to 3 sentences)
\item Stirring conclusion and invitation to read on...
\end{itemize}
\newpage

Databases and database technology have played an important part in both business and society for decades. Databases began as simple, monolithic servers. As the need for more powerful functionalities grew, things like relational databases and query languages (i.e. SQL) were invented to deal with the growing need for improved efficiency and ease of use. As the world became more connected and global, distributed databases emerged, and things like consensus algorithms and fault tolerance became popular topics in both academia and business.

Now, the world has become so interconnected that many different people and entities need to be able to use the same database(s). Traditional distributed databases typically assumed that all users were honest, and errors were the result of poor network conditions or other faults that were not due to adversarial behavior. In today's world, however, people who have competitive or even adversarial relationships with one another, even within the same entity, may need to access or edit the same information in the same database. To solve this problem, distributed ledger technology and blockchain technology were developed. The basic idea is fairly simple: with clever applications of cryptography and distributed systems concepts to traditional databases, many more useful applications can be constructed in ways that do not require a central trusted authority or reduce the trust requirements on the participants. With this in mind, we can view both blockchain and distributed ledger technology as the emerging field at the intersection of databases, cryptography and distributed systems.

Historically, databases have focused on single party applications purely out of necessity.  Since distributed databases allow for multiparty, shared database use, distributed ledgers can be equipped with multi-party business logic, which is more commonly referred to as \emph{smart contracts}.  This allows distributed ledgers to be used for substantially more applications than traditional databases.

While there are many definitions of the terms \emph{blockchain} and \emph{distributed ledger}, we will define them here for clarity. For the purposes of this paper, we refer to a \emph{blockchain} as a shared, append-only log of transactions (nothing can ever be erased or edited--only appends are allowed). We define a \emph{distributed ledger} as a multi-party, distributed database where there is no central trusted authority. When transactions are processed in blocks according to the ordering of a blockchain, the result is a distributed ledger.  In spite of our desire to clarify these terms, we will bow to popular press and use these terms interchangeably. 

Hyperledger builds on this rich technology background to bring blockchain-based distributed ledgers into a broad class of enterprise usages.  Broadly speaking, Hyperledger is an `umbrella' for open source distributed ledger platforms and related components and modules. The community of developers who participate in Hyperledger coordinate cross-industry, open source software development for projects that meet the diverse needs of those building and deploying distributed ledgers. 

The most popular existing blockchains like Bitcoin and Ethereum utilize completely trustless networks.  But most enterprise applications rely upon real world trust relationships that Hyperledger projects can leverage to gain efficiency, functionality, or both.

While supporting diversity of blockchain and distributed ledger technologies (necessary to meet the unique needs of enterprise applications) the consortium structure of Hyperledger also provides a means of bringing coordination from the chaos: identifying common components, avoiding duplication of effort, promoting interoperability and portability, and providing a diverse community for feedback.
\newpage

