One of the most exciting applications of blockchain is for self-sovereign identity: the idea that an individual owns their own ``identity'' and controls the data around it. 
This has profound implications for enterprise IT. 

\subsubsection{Indy extends traditional identity systems}
\textbf{Hyperledger Indy} is a distributed ledger with a primary focus on self-sovereign identity.
Indy shares some features with traditional enterprise identity systems such as 2FA, IDPs, LDAP,  OAuth, and so on, namely:
\begin{itemize}
\item Industrial-strength cryptography 
\item Rich metadata about identities
\item Sophisticated access control and policy 
\end{itemize}

But there is a profound difference: Indy identities are shared, not siloed and federated. 
An Indy identity is portable, so you can bring it with you wherever the distributed ledger is accepted. 

This means that 10 systems that support Indy identities don't create 10 separate identities for John Q. Public. 
Instead, all 10 systems access John's pre-existing identity on the blockchain. 
John can simply show up and use his identity. 
An organization can cancel John's access, but never his identity, because John owns it himself. 
And John---not the places that accept John's identity---controls access to his data.

\subsubsection{Indy fits enterprise IT better}
 Indy also shares some features with blockchain-based identity solutions such as Blockstack (\url{https://blockstack.org/}) and Uport (\url{https://www.uport.me/}). 
 All these technologies store identity on a distributed ledger to promote security and personal freedom. 

However, Blockstack depends on Bitcoin and and Uport on Ethereum. These proof-of-work (PoW) ecosystems impose a non-trivial cost on transactions, so that every new persona, public key rotation, published attribute, or pairwise relationship becomes a tangible expense. This creates a disincentive to pairwise relationships, which undermines privacy. 

Also, these ecosystems are global and public. They can't be special-purposed for a less-than-fully-global context. 

Indy, on the other hand, does not use PoW, so that transactions are free. 
And different instances of Indy can scale to fit whatever context is convenient. 
That makes Indy more flexible, more cost-effective, and more practical for managing identities through an enterprise blockchain.  
