Banks want to lend, but only to borrowers who are good risks. 
This motivates the banks to gather detailed, personally identifiable information (PII) from everyone who applies for a loan, such as date of birth, annual income, government ID or passport number, and so on. 

Ultimately, the banks use this PII to access an applicant's credit rating. 
Regulations may demand that certain PII is shared with authorities, for example, to prevent money laundering.
But retaining so much PII makes every bank a juicy target for hackers.

Seeking a loan isn't much fun for borrowers, either. 
The application process is intrusive, and it's  hard to ``shop around" for the best rates. 
Every new application multiplies the effort, and increases the risk that the applicant's PII will be abused.

\subsubsection{Hyperledger Indy can streamline this process}
\textbf{Hyperledger Indy} offers a transformative identity solution for this use case. 

With Indy, applicants can share only the information the banks need to make a decision, in a way that guarantees truth, builds confidence in the lender, and satisfies pressures from regulators. 

Anyone seeking a loan can apply to 100 different lenders in milliseconds, without placing any sensitive personal data into a hackable database.

Instead of disclosing any PII, loan applicants can generate zero-knowledge proofs that they are over 21, that their income on last year's taxes passed a certain threshold, that they hold a valid government ID number, and that their credit score met a certain threshold within the past week.

Strong, distributed ledger-based identity establishes a global source of truth, which delivers value to many parties. 
Applicants can give consent, and everyone can agree on when and how it was given.
Lenders can conform with regulations and show an immutable audit trail.

As a result, the market can operate more efficiently: Banks can offer loans with confidence, while applicants can effectively safeguard their PII.

\subsubsection{Other Hyperledger projects add further strengths}
This use case becomes even more compelling when you consider the added strengths of other Hyperledger projects.

For example, \textbf{Hyperledger Burrow} can turn loan applications into smart contracts, and attach them to strong identities as a seamless next step. 
And \textbf{Hyperledger Fabric} can drive a membership system by linking to the pre-existing, self-sovereign identity on the loan application. 
