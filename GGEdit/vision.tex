We live in a highly interconnected world. 
In the future, the world will no doubt become even more closely tied together.
In both our business and personal lives, more data, more digital content, more communication, and more sharing will be the norm. 
All this will require careful management of our security, privacy, and trust. 

\subsection{A common problem, and a sensible solution}
We expect to see a common problem: Many people will want to share data in a distributed database, but no single owner will be trusted by every user. 

The solution is distributed ledger technology (DLT). 
As data sharing increases, we expect blockchain technology and DLT to become more and more common.

But reaching widespread use of distributed ledgers will not be simple. 
For instance, gaining security and privacy with a blockchain often means sacrificing  performance. 
This suggests that we'll need a variety of different blockchains that can all communicate and interact seamlessly.
No one blockchain will work best for all applications.

The long-term vision for Hyperledger is driven by two main concerns: that the architecture must be modular and interoperable.

\subsection{Interchangeable modules}
We hope that eventually Hyperledger consists of many modules that can be assembled into a cohesive, functional, and secure distributed ledger. 
All these modules must be interchangeable with other modules of the same type. 
All modules must be able to communicate with other modules of the same or different types. 
And ideally, even a non-expert will be able to use them to set up a secure, interoperable blockchain quickly, easily, and efficiently.

\subsection{Many blockchains that work together}
We want to specifically point out that we do not believe any Hyperledger blockchain should be the ``one distributed ledger to rule them all.'' 
The Hyperledger community sees merit in many different blockchains. 
We hope that other developers consider interoperability with Hyperledger projects. 

\subsection{Not a single stack, but a collection of tools}
The goal for Hyperledger is not to become a single software stack. 
Instead, we want to create a collection of tools built with modularity and interoperability in mind. 
Then, any individual can use one, some, or all of the Hyperledger projects to create a distributed ledger to suit their needs.

In the future, we hope that Hyperledger can solve most of the common problems in the distributed ledger space. 
This will require a good community of developers, strong support from business and industry, and solid design principles. 
As shown in this paper, we have structured Hyperledger with all these in mind. 

