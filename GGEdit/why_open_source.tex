"Proprietary software" refers to a commercial product licensed by a vendor, normally for a fee. It's usually sold ``as is," giving buyers no way to add unique customizations or fix bugs. Proprietary software publishers carefully guard their ``source code"---the version that a programmer can read and edit---and distribute only the run-time version that's simply a long string of numbers. 

Open source is different. This is ``software that comes with permission to use, copy, and distribute, either as is or with modifications."\footnote{ Gartner. IT Glossary. Retrieved from \url{https://www.gartner.com/it-glossary/open-source}} Open source is usually free. Since the source code is  provided, developers are free to inspect, tweak, and improve the code---and to submit enhancements back. 

\subsection{Open source is popular and reliable}
When properly designed, coded, and deployed, open source is a proven and effective choice. 

For example, the Linux operating system runs 90 percent of the public cloud workload, has 62 percent of the embedded market share, and 99 percent of the supercomputer market share. \footnote{ 2017 State of Linux Kernel Development {https://www.linuxfoundation.org/2017-linux-kernel-report-landing-page/}}

The open source \textbf{Apache web server} has been the world's most popular web server for more than 20 years, and today supports more than 40\% of all active websites. \footnote{ February 2018 Web Server Survey. Netcraft. Retrieved from \url{https://news.netcraft.com/archives/2018/02/13/february-2018-web-server-survey.html}}

Other well-known open source software includes \textbf{mySQL}---the world's most popular database server--- and the \textbf{Firefox} browser. 

\subsection{Open source has some clear benefits}
According to 2015 and 2016 annual surveys of executives and developers,\footnote{ The 10th Annual Future of Open Source Survey. North Bridge and Black Duck Software. 2016. Retrieved from \url{https://www.blackducksoftware.com/2016-future-of-open-source}} these are the key reasons why enterprises choose open source software:  
\begin{itemize}
\item Competitive features and capabilities
\item No vendor lock-in, so customers can easily switch
\item High-quality solutions
\item The ability to customize and fix bugs, through access to source code
\item Lower total cost of ownership
\end{itemize}

Some years ago, the main attraction of open source was that it was ``free." 
Today, enterprises choose open source to reduce risk, gain speed-to-market, and get a competitive edge. 
Organizations want their programmers to focus on strategic projects that add significant value---such as adding industry-specific enhancements on top of a proven platform---rather than re-inventing the wheel. 

All these benefits are heightened when an enterprise confronts any profoundly new or challenging concept---like the web in years past---and like blockchain today. 
Rather than develop an entire infrastructure and engineer all of its own solutions, enterprises can ``stand on the shoulders" of others who already did pioneering work and freely shared it with the world. 

\subsection{Open source builds trust}
Blockchain represents a perfect opportunity to benefit from open source, since the concept of trust is woven deeply into all blockchain technologies. 

Blockchain systems are engineered to enable direct, peer-to-peer transactions between parties who don't fully trust one another, or don't trust any central authority to validate transactions or mediate disputes. 
Therefore, it's essential for these parties to trust in blockchain technologies. 

We believe that an open, collaborative approach that invites participation from all stakeholders is the most effective way to build trust for enterprises---enough trust for them to widely and rapidly adopt blockchain technologies. 

\subsection{Open governance}: 
Open Governance means that technical decisions -– which features to add, how to add them and when, among others – are made by a group of community-elected developers drawn from a pool of active participants. Participation in Hyperledger through becoming a Contributor and/or Maintainer is open to anyone.

As a developer or maintainer, this translates into one thing: trust. You know how decisions will be made and the process by which people will be selected to make these decisions. Hyperledger is vendor-neutral and technical contributions are based on meritocracy. 

Companies deploying blockchain internally, and those building products and services based on Hyperledger projects, tell us they trust Hyperledger because our technologies are built in the open by a broad community. 

\subsection{Open source promotes interoperability}
``Interoperable" means that a program can talk to programs from other organizations quickly and easily. In today's connected world, this is a must-have.
And in the future, we believe that many different blockchains will support many business processes for many organizations. 

Hyperledger eases interaction across numerous blockchains. The open source Hyperledger technologies are designed from the start to support interoperability across various blockchains.  

Hyperledger Quilt is expressly focused on facilitating cross-chain transactions.

\subsection{Open source makes sense for blockchain}
Both economics and common sense are on the side of a collaborative effort like Hyperledger. 

Enterprises need robust, feature-rich, modular blockchain platforms they can tailor to meet their requirements. 
Businesses as diverse as banks, car and airplane makers, and healthcare companies make a broad ecosystem of enterprises, all cooperating with the global Hyperledger developer community.

When many different users and vendors collaborate to co-create common technologies, everyone can enjoy the proven benefits including lower risk, higher quality, and faster time-to-market. 
We believe we can do more to advance blockchain technologies by working together than by working in isolation. 
