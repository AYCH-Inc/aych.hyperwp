As you know, ``proprietary software" is a commercial product licensed by a vendor for a fee. It's usually sold ``as is," giving buyers no way to add unique customizations or fix bugs. Commercial  software publishers carefully guard their ``source code"---the version that a programmer can read and edit---and distribute only the run-time version that's simply a long string of numbers. 

Open source is different. This is ``software that comes with permission to use, copy, and distribute, either as is or with modifications."\footnote{ Gartner. IT Glossary. Retrieved from \url{https://www.gartner.com/it-glossary/open-source}} Open source is usually free. Since the source code is  provided, developers are free to inspect, tweak, and improve the code---and to submit enhancements back to the non-profit group managing that software. 

\subsection{Open source is popular and reliable}
When properly designed, coded, and deployed, open source is a proven and effective choice. 

For example, the open source \textbf{Apache web server} has been the world's most popular web server for more than 20 years, and today supports more than 40\% of all active websites. \footnote{ February 2018 Web Server Survey. Netcraft. Retrieved from \url{https://news.netcraft.com/archives/2018/02/13/february-2018-web-server-survey.html}} 

\textbf{WordPress} is the world's most popular blogging platform, an open source platform used by 60\% of all websites with a content management system.\footnote{ Usage statistics and market share of WordPress for websites. W3Techs. Retrieved from \url{https://w3techs.com/technologies/details/cm-wordpress/all/all}} 

Other well-known open source software includes \textbf{mySQL}---the world's most popular database server---the \textbf{Firefox} browser, and the \textbf{Linux} operating system. 
It's no exaggeration to say that open source powers the web, the greatest example of open technology in history. 

\subsection{Open source has some clear benefits}
According to 2015 and 2016 annual surveys of executives and developers,\footnote{ The 10th Annual Future of Open Source Survey. North Bridge and Black Duck Software. 2016. Retrieved from \url{https://www.blackducksoftware.com/2016-future-of-open-source}} these are the key reasons why enterprises choose open source software:  
\begin{itemize}
\item Competitive features and capabilities
\item No vendor lock-in, so customers can easily switch
\item High-quality solutions
\item The ability to customize and fix bugs, through access to source code
\item Lower total cost of ownership
\end{itemize}

Some years ago, the main attraction of open source was that it was ``free." 
Today, enterprises choose open source to reduce risk, gain speed-to-market, and get a competitive edge. 
Organizations want their programmers to focus on strategic projects that add significant value---such as adding industry-specific enhancements on top of a proven platform---rather than re-inventing the wheel. 

All these benefits are heightened when an enterprise confronts any profoundly new or challenging concept---like the web in years past---and like blockchain today. 
Rather than develop an entire infrastructure and engineer all of its own solutions, enterprises can ``stand on the shoulders" of others who  already did pioneering work and freely shared it with the world. 

\subsection{Open source builds trust}
Blockchain represents a perfect opportunity to benefit from open source, since the concept of trust is woven deeply into the fabric of all blockchain technologies. 

Blockchain systems are engineered to enable direct, peer-to-peer transactions between parties who don't fully trust one another, or don't trust any central authority to validate transactions or mediate disputes. 
Therefore, it's essential for these parties to trust in blockchain technologies. 

We believe that an open, collaborative approach that invites participation from all stakeholders is the most effective way to build trust for enterprises---enough trust for them to widely and rapidly adopt blockchain technologies. 

\subsection{Three types of openness}
Any effective open source project needs three types of openness, and Hyperledger has been organized to promote all three: open governance, open development, and open review.

\textbf{Open governance}: 
Hyperledger embraces the transparency and openness of all Linux Foundation projects. 
The Linux Foundation provides the governing, legal, logistical, promotional, and technical structure that all software initiatives need. 

This open structure has attracted a growing number of contributors, hackfests, meet-up groups, and software projects to the Hyperledger umbrella. 
And the number of corporate members is rising steadily. 

\textbf{Open development}: 
Open source development is transparent, since the source code is freely available. Anyone can download the Hyperledger codebase and start contributing. The positions of authority in the community are determined in an open and democratic manner, as described in the \emph{Hyperledger Overview} available here: \url{https://www.hyperledger.org/resources/publications}.

\textbf{Open review}:
Open source invites audits, reviews, and comments from a diverse community of experts. 
This open review process is standard in the security and cryptography communities, where it helps make sure that systems are correctly planned and coded. 

Companies deploying blockchain internally, and those building products and services based on Hyperledger projects, tell us they trust Hyperledger because our technologies are built in the open by a broad consortium of users and vendors who regularly review the code to ensure its high quality. 

\subsection{Open source promotes interoperability}
``Interoperable" means that a program can talk to programs from other organizations quickly and easily. In today's connected world, this is a must-have.
And in the future, we believe that many different blockchains will support many business processes for many organizations. 

Without Hyperledger, achieving the needed interaction across numerous proprietary blockchains would be extremely difficult. 
Hidden problems, nasty surprises, and time-wasting ``gotcha's" would be common. 

The open source Hyperledger technologies are designed from the start to support interoperability across various blockchains. 
Hyperledger will also make it easier to port applications between different blockchains, saving time and effort for developers. 

With all this built-in operability, Hyperledger is ready for the blockchain-driven economy which is quickly developing. 

\subsection{Open source makes sense for blockchain}
Both economics and common sense are on the side of a collaborative effort like Hyperledger. 

Every enterprise need robust, feature-rich, modular blockchain platforms they can tailor to meet their exact needs. 
Businesses as diverse as banks, car and airplane makers, and healthcare companies make a broad ecosystem of enterprises, all cooperating under the Hyperledger umbrella.

When many different users and vendors collaborate to co-create common technologies, everyone can enjoy the proven benefits including lower risk, higher quality, and faster time-to-market. 
We believe we can do more to advance blockchain technologies by working together than by working in isolation. 
This fundamental belief led to the creation of Hyperledger. 
