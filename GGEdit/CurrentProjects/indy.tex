\textbf{Hyperledger Indy} is a distributed ledger, purpose-built for decentralized identity. 
Indy provides tools, libraries, and reusable components for creating and using independent digital identities rooted on blockchains or other distributed ledgers. 

These identities are interoperable across administrative domains, applications, and any other organizational ``silo.? Than means friends, competitors, and even antagonists can all rely on a shared source of truth. 

Indy answers fundamental questions such as, ``Who am I dealing with?" and ``How can I verify any data about the other party in this interaction?"�� 
Solid answers to these questions enable the trusted interactions that enterprises need.

\subsubsection{Key features of Indy}
\begin{itemize}
\item \textbf{Self-sovereignity}---Indy stores identity artifacts on a ledger with distributed ownership. 
These artifacts can include public keys, proofs of existence, cryptographic accumulators that enable revocation, and so on.
No one but the true owner can change or remove an identity. 
\item \textbf{Privacy}---By default, Indy preserves privacy, since every identity owner can operate without creating any correlation risk or breadcrumbs.
\item \textbf{Verifiable claims}---Identity claims can resemble familiar credentials such as  birth certificates, driver's licenses, passports, and so on. 
But these can be combined and transformed in powerful ways, using zero-knowledge proofs to enable selective disclosure of only the data required by any particular context.
\end{itemize}

\subsubsection{Many powerful benefits}
This combination of self-sovereignty, privacy, and verifiable claims is extremely powerful. 
Consider the many potential benefits. 

Bulk troves of sensitive data can vanish or become useless. 
The economics of hacking can be transformed, since less personally identifiable information (PII) is held by each business partner. 
The competing demands of preserving privacy and meeting regulations can be satisfied. 
Individuals and organizations can benefit from richer and more secure interactions. 
And the identity ecosystem can gain the innovation and dynamism of a free market. 

Despite the advanced cryptography under the hood, Indy's API is simple and straightforward. 
This API includes about 50 C-callable functions, with idiomatic wrappers for many mainstream programming languages. 

To find out more about Hyperledger Indy, see \url{https://github.com/hyperledger/indy-node} and further repositories named like \url{indy-*}.
