\textbf{Hyperledger Cello} is a blockchain module toolkit that aims to bring the on-demand ``as-a-service" deployment model to the blockchain ecosystem.
The goal is to help enterprises quickly and easily adopt blockchain technologies, by providing automated ways to create, manage, and terminate blockchains. 

Cello provides an efficient and automated multi-tenant chain service on top of various infrastructures, including bare metal, virtual machine, cloud platforms like Amazon Web Services (AWS), and container platforms like Docker Swarm and Kubernetes. 
All in all, this helps boost the efficiency of ``Blockchain as a Service (BaaS)." 

Cello also provides a real-time dashboard where users can: 
\begin{itemize}
\item  View the status of the blockchain system and see statistics such as blockchain events, chaincode performance, and system utilization
\item Manage blockchains (create, configure, and delete) and chaincode (deploy and upload private chaincode)
\end{itemize}

Hyperledger Cello currently supports \textbf{Hyperledger Fabric 1.0} as the main blockchain implementation. 
The project plans to support \textbf{Sawtooth} and other types of blockchains. 

The architecture follows the micro-service style, with pluggable implementations for most components. 
The main programming languages used are Python and JavaScript.

To find out more about Cello, see  \url{https://github.com/hyperledger/cello} and further repositories named like \url{cello-*}.
