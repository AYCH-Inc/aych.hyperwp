Hyperledger Sawtooth (\url{https://github.com/hyperledger/sawtooth-core} and further repositories named like \url{sawtooth-*}) is a modular platform for building, deploying, and running distributed ledgers. The Sawtooth design philosophy targets keeping distributed ledgers distributed and making smart contracts safe - particularly for enterprise use.

In fitting with this enterprise focus, Sawtooth is also highly modular. This enables enterprises and consortia to make policy decisions that they are best equipped to make.

At the time of this writing Sawtooth contains several technical innovations. Among these are:
\begin{itemize}
\item Unpluggable consensus - which goes beyond compile time pluggable consensus to allow a consortium to change consensus algorithms on a running blockchain simply by issuing a transaction.
\item PoET consensus - a new consensus algorithm with the scale benefits of Proof of Work but without its power drawbacks.
\item Transaction Families - a smart contract abstraction that allows users to write contract logic in the language of their choosing.
\item Ethereum contract compatibility - Transaction families also allow the integration of other contract interpreters including Hyperledger Burrow's Ethereum Virtual Machine. Permissioning, un-pluggable, consensus and other Sawtooth features allow an enterprise Ethereum configuration.
\item Parallel Transaction Execution - most blockchains require serial transaction execution in order to guarantee consistent ordering at each peer. Sawtooth includes an advanced parallel scheduler that splits blocks into parallel flows. Parallelism allows for faster block processing in order to partially address the performance drawback of blockchains compared to traditional databases.
\item Private Transactions - Clusters of Sawtooth nodes can be easily deployed with separate permissioning. This provides privacy and confidentiality among participants of that distinct chain. No centralized service leaks transaction patterns or other confidential information, however an intermediary such as Hyperledger Quilt would be required to connect disjoint chains. In the future Sawtooth plans to provide additional privacy and confidentiality features on top of Trusted Execution Environments and/or zero knowledge primitives.
\end{itemize}

Originally, Sawtooth was designed to explore scalability, security, and privacy questions prompted by the original distributed ledgers. That mandated a certain modularity that was lacking at the time. Starting from scratch allowed to employ lessons from those pioneering systems and branch into usages that the original currency ledgers weren't intended to address. PoET, the new consensus hits scalability, while Transaction Families, its contract logic, narrow the attack surface for contracts while simultaneously broadening the functionality. Its designers also have a keen interest in trusted execution environments and what role that can play in private transactions.

In branching into new business cases, it was felt important that the system preserve certain tenants of a distributed ledger. That is, in an enterprise deployment, the concept of a distributed ledger shouldn't collapse into a replicated database. Enterprise participants need autonomy and they have the right to run their own nodes. The set of participants will also be dynamic and the system – particularly consensus – must accommodate that volatility. It is not clear, for example, whether an O(n2) protocol with fixed membership like PBFT can support the scale or volatility of a distributed ledger at production levels. Further it seems inadvisable to sidestep the challenges of providing Byzantine Fault Tolerance and operate on only a Crash Fault Tolerant consensus. Finally, it is worth noting that, “public” and “private” define a spectrum of authorization policies – not a binary option for a distributed ledger.
