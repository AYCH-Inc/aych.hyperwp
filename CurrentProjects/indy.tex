Hyperledger Indy uses distributed ledger technology to make identity independent of organizational silos; friends, competitors, and even antagonists can all rely on a shared source of truth that answers fundamental questions such as, “Who am I dealing with?” and “How can I verify data about the other party in this interaction?” Solid answers to these questions enable the sort of trusted interactions demanded everywhere.

Because Indy stores identity artifacts (public keys, proofs of existence, cryptographic accumulators that enable revocation) on a ledger with distributed ownership, identities can be self-sovereign--nobody external to the identity owner can manipulate them or take them away. Identity in Indy is also privacy-preserving by default, meaning that an identity owner can operate without creating correlation risk or breadcrumbs.

A core technology for Indy is verifiable claims. These attestations of an identity’s attributes resemble credentials familiar to all of us: passports, driver’s licenses, birth certificates, and so forth. But they can be combined and transformed in powerful ways, using zero-knowledge proofs to enable selective disclosure of just those pieces of data that a particular context demands.

This combination of self-sovereignty, privacy, and verifiable claims is synergistic. Bulk troves of sensitive data vanish or become useless. The economics of hacking transform. The competing demands of privacy-preserving and strongly identifying regulations are satisfied. Individuals and organizations are free to seek mutual benefit from rich interaction; the identity ecosystem gains the innovation and dynamism of a free market. 

Despite the advanced crypto under the hood, Indy’s API is simple and straightforward. It consists of about 50 C-callable functions, with idomatic wrappers for many mainstream programming languages, 

