Hyperledger Fabric (\url{https://github.com/hyperledger/fabric} and further repositories named like \url{fabric-*}) is a platform for distributed ledger solutions, underpinned by a modular architecture delivering high degrees of confidentiality, resiliency, flexibility and scalability. It is designed to support pluggable implementations of different components, and to accommodate the complexity and intricacies that exist across the economic ecosystem.

Starting from the premise that there are no ``one-size-fits-all'' solutions, Fabric is an extensible blockchain platform for running distributed applications.  It supports modular consensus protocols, which allows the system to be tailored to particular use cases and trust models. Fabric runs distributed applications written in general-purpose programming languages, without systemic dependency on a native cryptocurrency.  This stands in sharp contrast to most other blockchain platforms for running smart contracts that require code to be written in domain-specific languages or rely on a cryptocurrency.  Furthermore, it uses a portable notion of membership for realizing the permissioned model, which may be integrated with industry-standard identity management.  To support such flexibility, Fabric takes a novel architectural approach and revamps the way blockchains cope with non-determinism, resource exhaustion, and performance attacks.

%Where Hyperledger Fabric breaks from some other blockchain systems is that it is private and permissioned. Rather than allowing anyone to be part of the network by either participating in the Proof-of-Work consensus or receiving transferrable forms of data such as tokens over the blockchain, the members of a Hyperledger Fabric network enroll through a membership services provider.

Fabric also offers the ability to create channels, allowing a group of participants to create a separate ledger of transactions. This is an especially important option for networks where some participants might be competitors and not want every transaction they make, such as a special price they’re offering to some participants and not others, known to every participant in the network. If a group of participants form a channel, then only those participants, and no others, have copies of the ledger for that channel.

