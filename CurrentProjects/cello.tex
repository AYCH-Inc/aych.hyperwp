Hyperledger Cello (\url{https://github.com/hyperledger/cello}) is an open framework to help people adopt blockchain technologies efficiently and easily, by providing automatic ways in blockchain provision and operational management. 

It brings the on-demand "as-a-service" deployment model to the blockchain ecosystem to reduce the effort required for maintaining the lifecycle of the Hyperledger blockchain frameworks. It provides a multi-tenant chain service efficiently and automatically on top of various infrastructures, including baremetal, virtual machine, Cloud platforms like AWS, and container platforms like Docker Swarm and Kubernetes, overall helping provide "Blockchain as a Service" efficiently. It also helps with maintainance through a dashboard where users can watch the statistics/status of the blockchain system (e.g., system utilization, blockchain events, chaincode performance), and manage the blockchains (e.g., create, config and delete) and chaincode (e.g., deploy and upload private chaincode) in real-time.

Hyperledger Cello currently has supported Hyperledger Fabric 1.0 as the main blockchain implementation, while it has plans to support more blockchain types like Sawtooth. The architecture follows the micro-service style, with pluggable implementations for most components. The main programing languages are Python and JavaScript.